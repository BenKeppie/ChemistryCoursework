\chapter {Evaluation}

\section{Equipment}

I used a volumetric pipette to get a specific measurement of a solution. The smallest division on the scale is 0.2 ml which leaves a +/- 0.1 ml uncertainty. 

I used weighing scales to measure out a desired amount of zinc. The smallest division on the scale is 3 decimal places, therefore that leaves a +/- 0.0005 g uncertainty; however as I take two measurements (weighing the weighing boat and then the zinc) the uncertainty would double to 0.001 g.

I used a thermometer to measure the temperature of the reaction. As the smallest division on the scale is 1 degrees, the uncertainty would be 0.5 degrees. As my measurements were from 25 degrees - 26 degrees, the uncertainty would be 24.5 degrees - 26.5 degrees.

\section{Problems}

I have had a number of problems which will have effected my experiment. These are discussed below.

\begin{enumerate}
\item Stock Solutions - As I was relying on the chemicals being given to me to be the correct concentration, I had no control, or knowledge of any incorrectly made solutions. This could effect my measurements either way as I was purely relying on being given the correct concentrations I asked for. Next time I would ensure that the stock solutions that I was using were freshly ordered in.

\item Zinc Clumping - I found that as the concentration of sulfuric acid increased, the tendency for the zinc to form clumps increased. This meant that a reduced surface area was available for a reaction to take place and therefore less hydrogen was being produced than what should've been. This meant that at higher concentrations the rate of reaction seems to be lower than it actually is. If I were to carry out this experiment again, I would use a magnetic stirrer in order to reduce the zinc powder from clumping.

\item Temperature - As I didn't control the temperature of the reaction, the reaction will have heated up which in itself caused an increase in the rate of reaction. This means that the hotter the day was, the faster the reaction seemed. If I were to carry out this experiment again I would carry the experiment out in a water bath so that I would control the experiments temperature.

\item Zinc Storage - As the zinc may have been stored in storage for a few years, zinc oxide may have formed around the surface area of the zinc powder. This means that some experiments that I carried out will have appeared to be slower than they should have been due to hydrogen gas not being produced. The equation for the reaction would be $ZnO + H_2SO_4 \rightarrow ZnSO_4 + H_20$ which, as it produces water, would dilute the acid more which would make the reaction even more slow. If I were to carry out this experiment again I would ensure that the zinc that I was using was freshly ordered in and not stored for a long period of time.

\item Bung Problems - When the bung is pushed into the conical flask, the pressure of the system increases which causes the gas syringe to falsely start moving. This causes a few ml of gas to be falsely collected and therefore over estimates the volume of hydrogen produced. If I were to carry out this experiment again I would use a gas tap so that there was no change in the pressure of the system when you close it. Another problem with the bung is the time it takes from adding the acid to the conical flask, to then placing the bung in the conical flask. Because of this, hydrogen gas is not collected straight away and therefore the volume of hydrogen collected is underestimated. If I were to carry this experiment out again I would have a chamber in which reactants can be added whilst the system is still closed.
\end{enumerate}